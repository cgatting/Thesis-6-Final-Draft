\documentclass[12pt]{article}


\usepackage[margin=1in]{geometry}
\usepackage[T1]{fontenc}
\usepackage{lmodern}
\usepackage{setspace}
\usepackage{microtype}
\usepackage{graphicx}
\usepackage{booktabs}
\usepackage{csquotes}\usepackage{hyperref}
\usepackage{doi}
\usepackage[nottoc]{tocbibind}
\usepackage[authoryear]{natbib}


\hypersetup{
  colorlinks=true,
  linkcolor=blue,
  citecolor=blue,
  urlcolor=blue
}

\title{Evolutionary Social Psychology: The Car as a Peacock Tail\\\large Automotive Semiotics, Costly Signals, and the Mating--Trust Tradeoff}
\author{Author Name}
\date{\today}

\onehalfspacing\begin{document}
\begin{titlepage}
  \centering
  \vspace*{1.2in}
  {\LARGE\bfseries Evolutionary Social Psychology: The Car as a Peacock Tail\par}
  \vspace{0.25in}
  {\Large Automotive Semiotics, Costly Signals, and the Mating--Trust Tradeoff\par}
  \vspace{0.75in}
  {\large Author Name\par}
  \vfill
  {\large \today\par}
\end{titlepage}

\pagenumbering{roman}
\tableofcontents
\newpage
\pagenumbering{arabic}

\section*{Full-page extract (key claim)}
\addcontentsline{toc}{section}{Full-page extract (key claim)}
Signals can be socially advantageous and socially costly at the same time because observers evaluate the same cue under different relational frames. A conspicuous automobile (large, powerful, or flashy) can operate as a \emph{costly signal} of resources and competitive orientation \citep{Veblen1899,Zahavi1975}, and in mating-relevant contexts such signals can be strategically aligned with short-term goals. Consistent with this logic, conspicuous consumption has been argued and shown to function as a sexual signaling system: it is interpreted as indicating short-term mating intent and tends to increase desirability primarily in short-term rather than long-term mating evaluations \citep{SundieEtAl2011}. Yet, the same conspicuousness that boosts perceived status can reduce perceived warmth and trust---a tradeoff emphasized in consumer research on the social costs of luxury \citep{DuboisEtAl2023}. In semiotic terms, cars are unusually effective carriers of such meanings because they are public, mobile signs embedded in advertising narratives and culturally shared codes \citep{Sorvali2014}. The result is a predictable ``mating--trust'' tension: the very properties that make a car legible as a bold, expensive display can simultaneously cue risk-taking, self-presentation, and low-investment intent, thereby undermining long-term trustworthiness judgments even when short-term interest rises.

\newpage

\maketitle

e

\begin{abstract}
Cars do more than transport bodies: they transport meanings. In many consumer cultures, vehicle choice is read as a public statement about status, taste, competence, and personality. This document synthesizes work in evolutionary social psychology and consumer research to argue that conspicuous automobiles can function as a \emph{costly signal}---a human analogue to the peacock's tail---and that the same signal may be strategically tuned to different social goals. In particular, showy and high-performance vehicles often convey resources and boldness, which can increase perceived short-term mating desirability, yet also invite countervailing inferences (e.g., self-presentation, dominance, or lower warmth) that may erode perceived long-term trustworthiness. We connect classic accounts of conspicuous consumption \citep{Veblen1899}, the handicap principle and costly signaling \citep{Zahavi1975}, and contemporary evidence that conspicuous consumption is interpreted as short-term mating intent and is advantageous primarily in that relational frame \citep{SundieEtAl2011}. We then integrate semiotic perspectives on automobility and car advertising as sign systems \citep{Sorvali2014} to clarify why cars are especially potent signals: they are durable, mobile, highly visible, and widely legible markers of identity.
\end{abstract}

\section{Introduction: Why Cars Invite Evolutionary and Semiotic Analysis}
The automobile is a paradoxical object in everyday life. It is at once utilitarian (a means of transport) and deeply symbolic (a means of self-presentation). In many societies, individuals do not merely \emph{own} cars; they \emph{perform} with them: arriving, departing, overtaking, parking, and idling in ways that are publicly observable. Even when the driver says they ``just like'' a particular model, observers routinely infer social information from the vehicle's size, brand, sound, condition, and modifications.

This inferential richness makes the car a natural meeting point for two traditions:
\begin{enumerate}
  \item \textbf{Evolutionary social psychology}, which asks how evolved motives (e.g., status seeking, affiliation, mating, threat management) shape social behavior and impression formation.
  \item \textbf{Automotive semiotics}, which treats cars as signs and investigates how meanings are encoded (by designers, advertisers, and cultures) and decoded (by audiences and everyday perceivers).
\end{enumerate}

The motivating thesis in both traditions is similar: the car is not only an object but a \emph{signal}. In the evolutionary vocabulary, conspicuous vehicles can serve as costly signals of resources, risk tolerance, or competitive intent. In the semiotic vocabulary, cars are culturally stabilized signifiers that bundle connotations (e.g., ruggedness, sophistication, ``family,'' ``performance'') into a portable artifact \citep{Sorvali2014}. The present essay develops the metaphor of ``the car as a peacock tail'' and uses it to explain a recurring empirical pattern: conspicuous vehicles may increase perceived short-term mating appeal while simultaneously harming perceived long-term trustworthiness.

\section{From Veblen to Zahavi: Conspicuous Consumption as a Signaling Problem}
\subsection{Veblen and the Social Logic of Visible Waste}
The idea that waste can be communicative is older than modern psychology. In \emph{The Theory of the Leisure Class}, Veblen argued that conspicuous consumption and conspicuous leisure are ways to display status through visible expenditure and non-productivity \citep{Veblen1899}. In this view, a purchase is not only a private utility-maximization decision; it is a public claim about one's place in a social hierarchy.

Cars are almost tailor-made for Veblenian display. They are expensive, externally visible, and difficult to hide; they also incur ongoing costs (fuel, insurance, maintenance). Because these costs are recurrent, the signal is not merely a one-time purchase but an ongoing demonstration of capacity to afford waste.

\subsection{The Handicap Principle and Costly Signaling}
Evolutionary biology reframes Veblen's intuition in terms of honest signaling. Zahavi's handicap principle proposes that a signal can be reliable precisely because it is costly; low-quality individuals cannot easily fake it \citep{Zahavi1975}. The peacock's tail is the canonical example: it is metabolically expensive and increases predation risk, yet it can be preferred by mates because it credibly indicates that the male can survive despite the handicap.

Translating this to humans, expensive or impractical consumption can be interpreted as a handicap that only certain individuals can sustain. A luxury sports car, for instance, may honestly signal wealth (ability to pay), social capital (access to status markets), or competitive orientation (willingness to invest in dominance displays). In social perception, such signals can influence how others allocate attention, deference, and romantic interest.


More recent consumer and evolutionary research formalizes and tests these intuitions by treating luxury purchases as \emph{costly signals} that can confer social benefits precisely because they are hard to fake \citep{NelissenMeijers2011}. In parallel, work on status consumption emphasizes that not all luxury signals are equally interpreted: the \emph{prominence} of a brand mark can change who uses it and what observers infer about the user \citep{HanNunesDreze2010}. These frameworks complement impression-management accounts \citep{Goffman1959} by explaining why a car can function as both a personal preference and a strategic public performance.

More recent consumer and evolutionary research formalizes and tests these intuitions by treating luxury purchases as \emph{costly signals} that can confer social benefits precisely because they are hard to fake \citep{NelissenMeijers2011}. In parallel, work on status consumption emphasizes that not all luxury signals are equally interpreted: the \emph{prominence} of a brand mark can change who uses it and what observers infer about the user \citep{HanNunesDreze2010}. These frameworks complement impression-management accounts \citep{Goffman1959} by explaining why a car can function as both a personal preference and a strategic public performance.


\section{Automotive Semiotics: Cars as Sign Systems}
Semiotic approaches emphasize that the meaning of a car is not inherent in its engineering alone; it is constructed through recurrent associations in advertising, media narratives, and everyday talk. A key point is \textbf{conventionality}: communities learn to read vehicle features as cues. The same physical property (e.g., high ground clearance) may connote safety and family protection in one frame and aggressive dominance in another.

One concrete research entry point is the analysis of car advertising as a multi-modal sign system. For example, \citet{Sorvali2014} analyzes verbal and pictorial signs in car advertisements using Peircean sign theory, highlighting how cars are embedded in culturally shared narratives and gendered meanings. Such semiotic work helps explain why the \emph{car} in particular, among many consumer goods, often becomes a privileged object for signaling. The car is public, mobile, and frequently encountered in competitive social ecologies (roads, city centers, nightlife districts), where comparison is easy and social attention is scarce.

\section{The ``Car as Peacock Tail'' Hypothesis}
The peacock tail analogy is not simply poetic; it is a compact model with testable implications. If cars function as mating-relevant costly signals, then:
\begin{enumerate}
  \item The propensity to display conspicuous cars should increase when mating motives are salient.
  \item Observers should infer mating intent from conspicuous car-related displays.
  \item The same display may have \emph{different} social payoffs depending on whether the observer is evaluating the sender as a short-term mate, a long-term mate, or a cooperative partner.
\end{enumerate}

These implications are consistent with a large body of work on conspicuous consumption as a sexual signaling system. The most directly relevant is \citet{SundieEtAl2011}, which links conspicuous consumption to short-term mating strategy and shows that women infer short-term intent from conspicuous displays; importantly, conspicuous purchasing enhances desirability in short-term but not long-term mating contexts.

While the original stimuli in such studies may include a variety of goods, cars are particularly salient in popular discussion because they are prototypically ``flashy'' and because vehicle choice is easily observed at the moment of social approach. A sports car arriving at a venue is a high-bandwidth signal; it is also comparatively difficult to counterfeit relative to smaller status items.

\section{Why Conspicuous Cars Can Raise Short-Term Mating Interest}
\subsection{Resource Display and ``Ability to Provide'' Heuristics}
In mate-choice psychology, resources are frequently discussed as a cue with potential relevance to parental investment and life outcomes. Even when individuals do not consciously endorse resource-based preferences, many cultures treat high spending as a proxy for success. A large, powerful, or luxury car can therefore act as a compressed representation of earnings, ambition, or access to capital.

From a signaling perspective, the car's cost does not merely imply wealth; it implies \textbf{disposable wealth}. If an observer believes that the vehicle is in excess of transportation needs, then the \emph{waste} becomes part of the message. This aligns with Veblen's logic and with costly signaling accounts, in which the signal's expense is precisely what makes it informative \citep{Veblen1899,Zahavi1975}.

\subsection{Risk, Boldness, and ``Good Genes'' Stories}
High-performance or modified vehicles also connote risk tolerance and sensation seeking. In some short-term mating contexts, boldness and dominance can be attractive because they imply competitive ability. The signal can therefore be interpreted not only as ``I can afford this'' but also ``I am the kind of person who takes chances.'' This is the ``high-risk, high-reward'' personality inference often discussed in popular and academic work on conspicuous displays.

However, these inferences are double-edged. The same traits that may be exciting in short-term contexts can be liabilities when evaluating stability, reliability, and cooperative intent.

\section{Why the Same Signal Can Harm Long-Term Trustworthiness}
\subsection{Frame-Dependent Evaluation: Mate Versus Partner Versus Cooperator}
Humans evaluate others under multiple relationship frames: potential short-term mate, potential long-term mate, friend, colleague, coalition partner, and so on. A central claim of this document is that conspicuous cars are \textbf{multi-meaning} signals whose evaluation depends on the frame.

\citet{SundieEtAl2011} provides a key empirical anchor for this argument: conspicuous consumption was associated with perceived short-term mating intent, and the desirability benefits were concentrated in short-term mating evaluations rather than long-term ones. If observers infer that a flashy car signals a low-investment mating strategy, then long-term partner desirability can decrease even if short-term interest increases.

\subsection{Warmth, Morality, and the Social Costs of Luxury Signals}Consumer research increasingly distinguishes between two broad dimensions of person perception: competence/status versus warmth/trust. Luxury signals often increase perceived status while reducing perceived warmth (and thus trust), producing a tradeoff in social outcomes. For example, recent work on luxury signaling discusses potential ``social costs'' alongside prestige benefits \citep{DuboisEtAl2023}.


This ``status--warmth'' structure is consistent with classic models of social cognition that treat warmth and competence as universal (and often competing) dimensions of impression formation \citep{FiskeCuddyGlick2007}. Put simply, a signal can raise perceived competence (``they must be successful'') while lowering perceived warmth (``they may not be kind or reliable''). Relatedly, research on pride distinguishes between achievement-based (authentic) pride and hubristic pride---a distinction that can matter for how displays of success are morally evaluated \citep{TracyRobins2007}. In car contexts, the same performance cue can be read as admirable mastery or as arrogant posturing, with downstream consequences for trust.

This ``status--warmth'' structure is consistent with classic models of social cognition that treat warmth and competence as universal (and often competing) dimensions of impression formation \citep{FiskeCuddyGlick2007}. Put simply, a signal can raise perceived competence (``they must be successful'') while lowering perceived warmth (``they may not be kind or reliable''). Relatedly, research on pride distinguishes between achievement-based (authentic) pride and hubristic pride---a distinction that can matter for how displays of success are morally evaluated \citep{TracyRobins2007}. In car contexts, the same performance cue can be read as admirable mastery or as arrogant posturing, with downstream consequences for trust.



Although the details vary across contexts and populations, the pattern is intuitive: a conspicuous display can be read as self-focused, competitive, or manipulative---attributes that conflict with cooperative expectations. This can be especially pronounced when the display appears \emph{loud} (highly visible, attention-seeking) rather than subtle.

\subsection{Strategic Self-Presentation and Suspicion of Motives}
Another route to reduced trustworthiness is not moral condemnation per se but \textbf{attributional suspicion}. If an observer believes that a flashy car was chosen \emph{to impress}, they may infer impression management, which in turn can trigger skepticism about authenticity and honesty. The signal becomes evidence not only of resources but of \emph{strategic social intent}.

In short-term mating contexts, strategic intent may be expected and even welcomed. In long-term contexts, the same intent can suggest opportunism.

\section{Integrating Semiotics and Evolutionary Psychology: A Unified Model}
An integrated account has three layers:
\begin{enumerate}
  \item \textbf{Material layer:} the car's observable features (brand, size, acceleration, sound, condition).
  \item \textbf{Semiotic layer:} culturally learned associations that map those features onto meanings (e.g., ``sports car'' $\rightarrow$ ``player''; ``SUV'' $\rightarrow$ ``family'' or ``dominance'').
  \item \textbf{Motivational layer:} evolved or chronically active goals that shape both signaling and interpretation (mating, status, coalition, safety).
\end{enumerate}

The same vehicle can therefore function as different signals depending on the audience and setting. A sports car outside a nightclub may be read primarily as a mating-oriented display; the same car in a workplace parking lot may be read as status competition or as misaligned priorities. Semiotics clarifies \emph{how} meanings are learned and stabilized; evolutionary psychology clarifies \emph{why} certain meanings (status, mating intent, risk) are socially consequential.

\section{Methodological Notes: What the Evidence Can and Cannot Show}
It is tempting to treat the ``car as peacock tail'' metaphor as a universal law, but empirical claims should be scoped carefully.
\begin{itemize}
  \item \textbf{External validity:} Many studies use vignettes, images, or brand cues rather than real-world longitudinal observation. They capture perception and intent inference, not necessarily actual relationship outcomes.
  \item \textbf{Heterogeneity:} Effects vary with age, socioeconomic context, gender norms, and local car cultures. In some settings, a large vehicle connotes practical needs; in others, it connotes dominance or environmental disregard.
  \item \textbf{Confounds:} Car choice correlates with many variables (income, geography, occupation). Experiments that manipulate the signal can isolate causal perception effects, but field data require careful controls.
\end{itemize}

Despite these cautions, a convergent picture emerges: conspicuous displays can be adaptive in specific social games and costly in others. That is exactly what signaling theory predicts.

\section{Implications}
\subsection{For Individuals: Signaling Strategy and Audience Design}
If the goal is short-term attraction, the literature suggests that conspicuous signals may sometimes help because they broadcast status and a low-investment mating orientation \citep{SundieEtAl2011}. If the goal is long-term partnership or cooperative trust, conspicuous displays may backfire by reducing warmth or inviting suspicion.

The practical implication is not ``never buy a flashy car'' but ``treat vehicle choice as audience-dependent communication.'' In semiotic terms, a car is an utterance; in evolutionary terms, it is a strategic signal.

\subsection{For Marketers: Segmenting by Relational Frame}
Automotive marketing often implicitly segments consumers by identity narratives: ruggedness, performance, family safety, eco-consciousness, luxury. The present framework recommends segmenting by \emph{relationship goals} as well: short-term mating contexts versus long-term family contexts produce different desirability criteria and different inferences.

\subsection{For Researchers: Open Questions at the Intersection}
Several questions remain underexplored:
\begin{enumerate}
  \item How do observers integrate car signals with contradictory cues (e.g., an expensive car paired with conspicuously prosocial behavior)?
  \item Do ``quiet luxury'' cues in cars (subtle premium models) maintain status benefits without warmth costs?
  \item How do electric vehicles reconfigure the semiotic landscape by adding moral and environmental meanings to status meanings?
  \item Can longitudinal designs link car signaling to real relationship duration, commitment, and reputational outcomes?
\end{enumerate}

\section{Conclusion}
The ``car as a peacock tail'' metaphor captures a real social dynamic: conspicuous vehicles can operate as costly signals that broadcast resources and boldness, and observers often interpret such signals through mating-relevant and status-relevant lenses. Yet signaling gains are context-sensitive. Evidence on conspicuous consumption as a sexual signaling system indicates that showy displays can increase desirability in short-term mating frames while failing to help---and potentially harming---evaluations in long-term frames \citep{SundieEtAl2011}. Semiotic analysis explains why cars are especially readable signs and why their meanings are culturally amplified through advertising and shared narratives \citep{Sorvali2014}. Taken together, evolutionary social psychology and automotive semiotics suggest a nuanced view: conspicuous cars are neither universally advantageous nor merely frivolous; they are communicative tools whose benefits and social costs depend on what relationship game the audience believes is being played.

\bibliographystyle{apalike}
\bibliography{references}

\end{document}
